% CREATED BY DAVID FRISK, 2015

\chapter{Governing Equations and Solution methods}

\section{Governing Equations}
\subsection{Heat equation}The unsteady heat equation in 1D solved in this code is given below,
\begin{equation}\label{eq:uhe}
\frac{\partial T}{\partial t} = \alpha \frac{\partial^2 T}{\partial x^2}
\end{equation}

$T$ is temperature and $x$ and $t$ represent the spatial and temporal coordinates respectively. Here Thermal Diffusivity $\alpha$ is a constant and is set to $1.9\times 10^-5\hspace{0.2cm} m^2/s$.

\subsection{Heat equation for Method of manufactured solutions(MMS)}
Consider a manufactured solution for $T$,
\begin{equation}\label{c2}
    T(x,t)=e^{-t} sin(\pi x)
\end{equation}
Applying the manufactured solution to the governing equation,
\begin{equation}\label{c3}
\begin{gathered}
    \frac{\partial T}{\partial t}=-e^{-t} sin(\pi x)\\
    \frac{\partial^2 T}{\partial x^2}=-e^{-t} \pi^2 sin(\pi x)
\end{gathered}
\end{equation}
Using equations (\ref{eq:uhe} and \ref{c3}) the obtained source term is,
\begin{equation}\label{c4}
Q=e^{-t}sin(\pi x) (\alpha \pi^2-1)
\end{equation}
Using (\ref{c4}) the governing equation with (\ref{c2}) as the solution is written as,
\begin{equation}\label{c5}
\frac{\partial T}{\partial t}-\alpha\frac{\partial^2 T}{\partial x^2}=Q
\end{equation}






\section{Numerical Method}
\hspace{0.25cm}The governing equation can be in general represented as,
\begin{equation}\label{eq:210}
   \frac{\partial Q}{\partial t}+\frac{\partial F_j}{\partial x_j}=Q_s
\end{equation}

Here, $Q$, $F$ and $Q_s$ represent the unsteady term ,flux term and the source term respectively. 

For unsteady heat equation,
\begin{equation}
   Q=\begin{bmatrix}
   T
   \end{bmatrix},F=\begin{bmatrix}
    -\alpha\frac{\partial T}{\partial x_j}\\
   \end{bmatrix}
\end{equation}

\subsection{Finite Volume Method}
Equation (\ref{eq:210}) is integrated over an arbitrary Control Volume  V,
\begin{equation}\label{eq:int}
    \int_V\frac{\partial Q}{\partial t}\,dV +\int_V\frac{\partial F_j}{\partial x_j}\,dV=\int_VQ_s\,dV
\end{equation}
Further Gauss theorem is applied to the flux part thereby solving it over the Control surface  and cell average 
$\overline{Q}$ is used to represent the unsteady term.

\begin{equation}\label{eq:gc}
    \frac{\partial \overline{Q}}{\partial t}V+ \int_S  F_j.\,dA_j = \overline{Q_s} V
\end{equation}
The surface integral in equation (\ref{eq:gc}) can be further approximated using the areas of the control surfaces and the average face flux.
\begin{equation}\label{eq:ahf}
    \frac{\partial \overline{Q}}{\partial t}V +  \sum\limits_{i=1}^{faces}  F_j^i.\,A_j^i = \overline{Q_s} V \hspace{1cm} A_j=n_j A
\end{equation}



The equation (\ref{eq:ahf}) is further integrated over time $t$ for temperature $T$,
\begin{equation}\label{eq:atime}
   \int_t \frac{\partial \overline{T}}{\partial t}V\,dt = \alpha \int_V \sum\limits_{i=1}^{faces}  \frac{\partial T}{\partial x_j}^i\,A_j^i\,dt+\int_t \overline{Q_s}V\,dt 
\end{equation}
The equation is averaged over time,

\begin{equation}\label{eq:atimen}
   (\overline{T}^{n+1}-\overline{T}^n)V = \alpha \int_V \sum\limits_{i=1}^{faces}  \frac{\partial T}{\partial x_j}^i\,A_j^i\,dt+\int_t \overline{Q_s}V\,dt 
\end{equation}
Further first order Euler explicit method which is a forward time marching scheme is used as the time scheme,
\begin{equation}\label{eq:final}
   \overline{T}^{n+1} =\overline{T}^n+ \frac{\alpha \Delta{t} }{V}  \sum\limits_{i=1}^{faces}  (\frac{\partial T}{\partial x_j}^i)^n\,A_j^i+\overline{Q_s} \Delta{t}
\end{equation}

Equation (\ref{eq:final}) is further reduced to its 1D form which essentially can be seen as finite difference method discretization. Further, Central differencing is applied for the flux term.

    


