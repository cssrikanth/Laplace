% CREATED BY DAVID FRISK, 2015
%An Informative Headline describing the Content of the Report\\
%A Subtitle that can be Very Much Longer if Necessary\\
%NAME FAMILYNAME\\
%Department of Some Subject or Technology\\
%Chalmers University of Technology \setlength{\parskip}{0.5cm}

\thispagestyle{plain}            % Supress header 
\setlength{\parskip}{0pt plus 1.0pt}
\section*{Abstract}
\hspace{0.25cm}Grid distortion is usually one of the major problems in the field of Computational Fluid dynamics (CFD). The quality of the results obtained from the numerically modeled simulations hugely depends on grid quality. There are many instances in modeling a fluid dynamics problem where grid distortion is inevitable. When this occurs the corresponding Numerical Scheme used to approximate the fluxes must account for these opposing changes. \\
\\
\hspace{0.25cm}A new numerical scheme called the Preferred Direction Diffusion Scheme for the calculation of diffusive fluxes is presented in this thesis. The numerical scheme proposed here is less sensitive to grid quality, therefore, the transformation of the grids is expected to be more accurate compared to the traditional transformation techniques like central differencing.\\
\\
\hspace{0.25cm}Initially, the scheme is implemented in MATLAB to study its Mathematical behavior. It is further applied to an Unsteady heat Conduction equation in 3D. It is tested on a simple case-a Square duct with adverse grid conditions. The obtained results are compared with the results obtained from an existing numerical scheme with conventional transformation technique. Later, both the results are compared to an analytical solution and conclusions are drawn based on that. \\
\\
\hspace{0.25cm}The implemented scheme is further evaluated for its robustness and accuracy through appropriate code verification techniques. Code verification usually involves error evaluation of the numerical schemes for known benchmark results. The obvious choice for a benchmark solution is the analytical solution but it's usually impossible to obtain them with a sufficient solution structure. The method of Manufactured solutions (MMS) plays a useful role in this situation. This method is fairly straightforward and purely a mathematical procedure. \\
\\
\hspace{0.25cm}MMS is implemented in CALC++ for an unsteady heat equation in 3D and Navier-stokes equation. CALC++ is a massively parallel incompressible flow solver (written in C++) developed at the division of Fluid Dynamics, Chalmers University of Technology. The scheme is evaluated by subjecting it to systematic grid tests. Its performance is judged by studying the rate of reduction of the discretization error with increasing grid resolution.

% KEYWORDS (MAXIMUM 10 WORDS)
\vfill
Keywords: CFD, Numerical scheme, Diffusive fluxes, PDS, CD, MMS, CALC++, MATLAB .

\newpage                % Create empty back of side
\thispagestyle{empty}
\mbox{}