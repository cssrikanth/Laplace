% CREATED BY DAVID FRISK, 2015
\chapter{Directions to use the code}
\section{General instructions}
The following points directs the user on how to compile and use the 1D Fortran code.
\\

\begin{itemize}
  \item Unzip the folder contents.
  \item The user must have already installed Fortran and Python compilers.
  \item Consult the Makefile for general compiling information.
  \item Type "make clean" to remove any already compiled files.
  \item Open "global.f95" file.
  \item Enter the details required to set up the case in that particular file.
  \item Type "make" to compile.
  \item Type "./heat" to execute the case.
  \item Text files are generated in the same folder.
  \item To post process, type "make output\_script.py".
  \item A list of text files available to post process is generated in the terminal.
  \item Choose for which time the user wants to display the plot.
  \item Type that particular time(the entire number) without the .txt extension.
  \item The plot is saved in the folder which can be viewed.
  \item To view another plot type "yes" and enter the time.
  \item To run another case with the compiled code type "make remove", make the changes in the global file and then type "make.
  \item To view the MMS results, change the value in the global file to "1"
  \item MMS results can be directly compared with the analytical results(for MMS) in the terminal.
\end{itemize}

\section{To note}

\begin{itemize}
  \item The code has some minor bugs but it should work as long as the stability criterion for explicit Euler is met.
  \item MMS is showing stability issues with the code which is natural due to the usage of explicit Euler. 
  \item For this reason, more work has to be done to better the MMS implementation but the working can be generally understood.
  \item It has to be considered that this a very small implementation and a base for improving it's efficiency. 
  
  
\end{itemize}

